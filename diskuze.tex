\section*{Diskuze}
Teoretické hodnoty $\omega_3$ a $\omega_4$ podle vztahu \eqref{eq::omega34} jsou pro $l =$ \SI{27.8(1)}{cm}:
\begin{itemize}
\item \begin{tabbing} Pružina A: \= $\omega_3=\SI{3.52(3)}{\per\s}$	\\ \> $\omega_4=\SI{0.15(3)}{\per\s}$  \end{tabbing}
\item \begin{tabbing} Pružina B: \= $\omega_3=\SI{3.44(3)}{\per\s}$	\\ \> $\omega_4=\SI{0.08(3)}{\per\s}$  \end{tabbing}
\end{itemize}

Všechny naměřené hodnoty se velmi přesně shodují s teoretickými, až na $\omega_3$ s pružinou A (naměřeno \SI{2.96(3)}{\per\s}), která se od teoretické výrazně liší.
Vzhledem k tomu, že naměřené $\omega_4$ velmi přesně odpovídá naměřeným $\omega_1$ a $\omega_2$, je nejpravděpodobnější příčinou hrubá chyba při měření periody $T_3$ (pravděpodobně chybný odečet počtu měřených period).
Nemůžeme ale vyloučit, že se o hrubou chybu nejedná a pohyb kyvadel našemu modelu neodpovídá.
V každém případě je měření neprůkazné a mělo by se zopakovat.

