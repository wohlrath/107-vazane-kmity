\section*{Výsledky měření}
Nejdříve jsme změřili periodu obou kyvadel, když nebyly vázány, abychom se ujistili, že skutečně kmitají stejně.
Naměřené $T_0$ a odpovídající $\omega_0 = 2\pi / T_0$ jsou uvedeny v tabulce \ref{tab::nevazany}.

\begin{tabulka}[htbp]
\centering
\begin{tabular}{ccc}

kyvadlo & $T_0$ (\si{\s}) & $\omega_0$ (\si{\per\second})  \\ \hline
levé & \num{1.88(2)} & \num{3.35(3)} \\
pravé & \num{1.88(2)} & \num{3.35(3)} \\
\end{tabular}
\caption{Kmity nevázáných kyvadel}
\label{tab::nevazany}
\end{tabulka}

Časy $T_1$, $T_2$, $T_3$ a $T_S$ jsme změřili s dvěma pružinami A ($k=$~\SI{7}{\newton\per\metre}) a B ($k=$~\SI{4}{\newton\per\metre}) při $l=$~\num{27,8(1)}\footnote{Číslo v závorce označuje standardní odchylku v řádu poslední uvedené cifry, např. \num{0.50(12)} znamená hodnotu \num{0.50} se standardní odchylkou \num{0.12}.}\si{\cm}
Spolu s odpovídajícími $\omega_1$, $\omega_2$, $\omega_3$ a $\omega_4$ a vypočtenými $\kappa$ jsou uvedeny v tabulce \ref{tab::vsechnyomegy}.

\begin{tabulka}[htbp]
\centering
\begin{tabular}{cccccccccc}

& $T_1$ (\si{\s}) & $\omega_1$ (\si{\per\second})  & $T_2$ (\si{\s}) & $\omega_2$ (\si{\per\second}) & $T_3$ (\si{\s}) & $\omega_3$ (\si{\per\second}) & $T_S$ (\si{\s}) & $\omega_4$ (\si{\per\second}) & $\kappa$ \\ \hline
% pr & T1 & w1 & T2 & w2 & T3 & w3 & TS & w4 & k \\
A & \num{1.87(2)} & \num{3.37(3)} & \num{1.71(2)} & \num{3.67(4)} & \num{2.12(2)} & \num{2.96(3)} & \num{21.2(3)} & \num{0.148(2)} & \num{0.087(13)} \\
B & \num{1.87(2)} & \num{3.36(4)} & \num{1.79(2)} & \num{3.51(3)} & \num{1.83(2)} & \num{3.43(4)} & \num{40.4(3)} & \num{0.0777(6)} & \num{0.045(13)} \\
\end{tabular}
\caption{Kmity vázaných kyvadel při různých počátečních podmínkách}
\label{tab::vsechnyomegy}
\end{tabulka}

S pružinou A jsme navíc změřili $T_1$ a $T_2$ pro různé vzdálenosti upevnění pružiny $l$. Spolu s odpovídajícími $\omega_1$, $\omega_2$ a vypočtenými $\kappa$ jsou uvedeny v tabulce \ref{tab::pruzinaAruznyl}.
Závislost $\kappa(l)$ je vynesena do grafu \ref{grp::kappanal}.

\begin{tabulka}[htbp]
\centering
\begin{tabular}{cccccc}

$l$ & $T_1$ (\si{\s}) & $\omega_1$ (\si{\per\second})  & $T_2$ (\si{\s}) & $\omega_2$ (\si{\per\second}) & $\kappa$ \\ \hline
% l & T1 & w1 & T2 & w2 & k \\
\num{27.8(1)} & \num{1.87(2)} & \num{3.37(3)} & \num{1.71(2)} & \num{3.67(4)} & \num{0.087(13)} \\
\num{24(1)} & \num{1.87(2)} & \num{3.36(3)} & \num{1.75(2)} & \num{3.57(3)} & \num{0.066(11)} \\
\num{19(1)} & \num{1.87(2)} & \num{3.36(3)} & \num{1.80(2)} & \num{3.50(3)} & \num{0.041(11)} \\
\num{14(1)} & \num{1.87(2)} & \num{3.36(3)} & \num{1.83(2)} & \num{3.43(3)} & \num{0.022(11)} \\
\num{9(1)} & \num{1.88(2)} & \num{3.35(3)} & \num{1.86(2)} & \num{3.39(3)} & \num{0.011(11)} \\
\end{tabular}
\caption{Kmity kyvadel vázaných pružinou A upevněnou v různých vzdálenostech od uložení závěsu}
\label{tab::pruzinaAruznyl}
\end{tabulka}

\begin{graph}[htbp] 
\centering
\input{graf.tex}
\caption{Závislost stupně vazby $\kappa$ na vzdálenosti upevnění pružiny A od uložení závěsů kyvadel $l$}
\label{grp::kappanal}
\end{graph}

